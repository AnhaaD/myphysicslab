\documentclass[final]{article}
%\usepackage[final]{graphics}
% final means "include the picture".  draft means "put a space with filename"
\usepackage[final]{graphicx}
\usepackage{latexsym}
\usepackage{amssymb}
\usepackage{amsmath}
\usepackage[normalem]{ulem}
% layout package provides the \layout command that shows page layout settings
\usepackage{layout}
% hyperref[1] provides ability to create hyperlinks
\usepackage{hyperref}
% framed provides an easy way to box a paragraph within a document
\usepackage{framed}
%\usepackage{showkeys}  %shows labels in margins see Gratzer p. 249
% these settings are mostly to make the margins smaller
\abovecaptionskip=0pt
\oddsidemargin=0pt
\textwidth=450pt
\textheight=650pt
\topmargin=-0.5in
\setlength{\marginparwidth}{0pt}
%  A tilde ?~? character generates a non-breaking space

\setlength{\parindent}{0cm} % no indent for paragraphs
\setlength{\parskip}{1.5ex} % more space between paragraphs.
\title{Path Length of a Parabola}
\author{Erik Neumann\\
erikn@myphysicslab.com}
\date{February 3, 2017}
\begin{document}

\maketitle

We seek an expression for the path length of a parabola $y = x^{2}$ starting from $x = -1$.

The length of a plane curve is given by

\begin{equation}
  L_{a}^{b} = \int_{a}^{b} \sqrt{ 1 + \left( \frac{dy}{dx} \right)^{2} } dx
\end{equation}

For the parabola, we have:

\begin{equation}
  \frac{dy}{dx} = 2x
\end{equation}

and so we can write the path length starting at $x = -1$ as a function $p(x)$:

\begin{equation}
  p(x) = \int_{-1}^{x} \sqrt{ 1 + (2t)^{2} } dt
\end{equation}

Make a substitution $u = 2t$ and $du = 2 dt$.  The limits of integration also change accordingly.

\begin{equation}
  p(x) = \int_{-2}^{2x} \sqrt{ 1 + u^{2} } \frac{du}{2}
\end{equation}

\begin{equation}
  p(x) = \frac{1}{2} \int_{-2}^{2x} \sqrt{ 1 + u^{2} } du
\end{equation}

This is a standard integral for which we can find the solution for in a table of integrals.

\begin{equation}
  p(x) = \left. \frac{1}{2} \left( \frac{u}{2} \sqrt{ 1 + u^{2} } +\frac{1}{2} sinh^{-1} u \right) \right]_{u=-2}^{u=2x}
\end{equation}

\begin{equation}
  p(x) = \left. \frac{1}{4} \left( u \sqrt{ 1 + u^{2} } + sinh^{-1} u \right) \right]_{u=-2}^{u=2x}
\end{equation}

We can write this out as

\begin{equation}
  p(x) = \frac{1}{4} \left( ( 2x \sqrt{ 1 + 4x^{2} } + sinh^{-1} (2x) ) - (-2 \sqrt{1 + (-2)^{2} } + sinh^{-1}(-2)) \right)
\end{equation}

\begin{equation}
  p(x) = \frac{1}{4} \left(  2x \sqrt{ 1 + 4x^{2} } + sinh^{-1} (2x)  +2 \sqrt{5} - sinh^{-1}(-2) \right)
\end{equation}

for $x >= -1$.

Note that $p(-1) = 0$ as required for the path length.

For reference, here is a formula for finding hyperbolic inverse sine using logarithms, which is valid for any real $u$:

\begin{equation}
  sinh^{-1}(u) = ln(u + \sqrt{1 + u^{2} })
\end{equation}

To find radius of curvature: it is inverse of $\kappa$ given by:

\begin{equation}
  \kappa = \frac{|\frac{d^2 y}{dx^2}|} {(1 + (\frac{dy}{dx})^2)^{\frac{3}{2}}}
\end{equation}

For the parabola:

\begin{equation}
  \kappa = \frac{2} {(1 + (2x)^2)^{\frac{3}{2}}}
\end{equation}



\end{document}
